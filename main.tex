\documentclass[12pt]{article}
\usepackage{listings}
\usepackage{graphicx}
\usepackage{amsmath}
\usepackage{hyperref}
\begin{document}
\begin{titlepage}
   \begin{center}
       \vspace*{1cm}
       \large
       \textbf{Project 2: Feature Detection and Matching}
       \normalsize

       \vspace{0.5cm}

       Author: Gabriel Hofer

       \vspace{0.5cm}

       CSC-414 Introduction to Computer Vision

       \vspace{0.5cm}

       Instructor: Dr. Hoover

       \vspace{0.5cm}

        March 31, 2020

       \vfill

       Computer Science and Engineering\\
       South Dakota School of Mines and Technology\\
   \end{center}
\end{titlepage}
%------------------------------------------------------------------------------------
\newpage
\small
\textbf{Step 1: Feature Extraction}\\ 

I implemented the Harris Corner Detection algorithm for feature extraction. \\

\textbf{Step 2: Descriptors}\\

For each Keypoint, K, we create a 16x16 window around centered at K.
Then, we partition this 16x16 window into 16 4x4 sub-windows.
For each 4x4 sub-window, we compute and apply a Laplacian filter to the image:
\[
    L(x,y,\sigma) = G(x,y,\sigma) * I(x,y)
\]
For each cell in the 4x4 subwindow, we compute the magnitude, $m(x,y)$, and orientation
$ \theta (x,y) $ of the cell with respect to the center of the window.
\[
    m(x,y) = \sqrt{(L(x+1,y)-L(x-1,y))^2+(L(x,y+1)-L(x,y-1))^2}
\]
\[
    \theta (x,y) = \arctan{(L(x,y+1)-L(x,y-1))/(L(x+1,y)-L(x-1,y))}
\]
An 8-bin histogram is constructed to determine the orientation of the whole 4x4
subwindow. Each cell is assigned to one of the bins based on it's orientation ($\theta$).
The magnitude of the cell is added to its bin. 
The orientation of the histogram is determined by the largest bin in the histogram.
Once a histogram has been made for each of the 16 4x4 subwindows, a histogram of 
oriented gradients is made for the 16x16 window using the results from the 4x4 subwindows.
Specifically, each 4x4 histogram acts like a single cell in the 16x16 window.
the magnitude each 4x4 window is added to the appropriate bin in the histogram based
on the orientation of the 4x4 window. 
Once again, the orientation and mangitude of the 16x16 window is determined by the 
largest bin in the histogram.

\textbf{Step 3: Feature Matching}\\

My feature matching algorithm looks at all pairs of feature descriptors: $(d0,d1)$.

A pair becomes a match if it meets two criteria:
\begin{enumerate}
    \item the difference of orientations between d0 and d1 is less than the threshold
    \item the difference of magnitude between d0 and d1 is less than the threshold
\end{enumerate}

For drawing the matches between the two descriptors I used a 
\href{https://en.wikipedia.org/wiki/Digital_differential_analyzer_(graphics_algorithm)}{digital differential analyzer (DDA)} 
algorithm.

\end{document}


